% Copyright 2020 by Junwei Wang <i.junwei.wang@gmail.com>
%
% This file may be distributed and/or modified under the
% conditions of the LaTeX Project Public License, either version 1.3c
% of this license or (at your option) any later version.
% The latest version of this license is in
%   http://www.latex-project.org/lppl.txt

% \documentclass[aspectratio=169,compress]{beamer}
\documentclass[aspectratio=169,compress]{beamer}

\usepackage[english]{babel}
\usepackage{metalogo}
\usepackage{listings}
\usepackage{fontspec}
\usepackage{tikz}

% \usetheme{Nord}
\usetheme[style=light]{Nord}


%\usepackage[spanish, es-tabla]{babel}
\usepackage[utf8]{inputenc}
\usepackage{hyperref}




\setmainfont{Yanone Kaffeesatz}
%\setsansfont{Andika New Basic}
\setmonofont{DejaVu Sans Mono}

\setbeamerfont{frametitle}{parent=structure,size=\Large}

\AtBeginSection[]
{
  \begin{frame}[c,noframenumbering,plain]
    \tableofcontents[sectionstyle=show/hide,subsectionstyle=show/show/hide]
  \end{frame}
}

\AtBeginSubsection[]
{
  \begin{frame}[c,noframenumbering,plain]
    \tableofcontents[sectionstyle=show/hide,subsectionstyle=show/shaded/hide]
  \end{frame}
}

\title{Arquitecturas y Organización de Computadoras I}
\subtitle{1: Abstracciones en la computadora y tecnología}
\author{Rafael Ignacio Zurita}
\institute{Depto. Ingeniería de Computadoras}
\date{\today}

\begin{document}

\begin{frame}[plain,noframenumbering]
\bigskip
  \maketitle
\end{frame}

% video
% mostrar varias computadoritas
% mostrar placa con integrados y pcb
% mostrar foto de chip y hablar de los transistores
% mostrar imagen wakerly de transistor CMOS

% imagenes 
% foto de performance
% cuadro python vs C
% foto eras tecnologicas

% historia de ibm
% fotos de computadoras hitos
%     ibm 360 (mainframe), cray1 (supercoputaodra)
%     pdp-11 (creacion de unix) (minicomputadora)
%     4004 primer chip integrado
%     apple II y ibm pc (computadora personal)
%     open mobile komunications (smartphones)
%     iphone smartphone

\section{Abstracciones en la computadora y tecnología}

\subsection{Eras tecnológicas}
\subsection{Limitaciones tecnológicas}
\subsection{Tiempo de ejecución (rendimiento)}
\subsection{Terminología: Arquitectura y Organización de un procesador}
\subsection{Avances (cuadro)}
\subsection{MIPS/RISCV ISA (Arquitectura de una computadora real)}

\subsection{Abstracciones en la computadora}

\begin{frame}{La computadora: Un sistema complejo}{CHIP}

    \begin{columns}[onlytextwidth,T]
      \column{\dimexpr\linewidth-60mm-5mm}

	\begin{itemize}
	\begin{small}
\bigskip
  \item[Chip] Die en inglés, es empaquetado dentro de un componente que permite su utilizacióin mecánica en un PCB.

\bigskip
\item[Densidad] La tecnología que se utiliza para fabricar los chips (dies), en los circuitos integrados, es el trasistor CMOS.\\
Actualmente la densidad es tan grande que existen miles de millones de transistores en un unico chip.

	\end{small}
	\end{itemize}

      \column{60mm}
    \includegraphics[width=80mm]{images/chip.png}

    \end{columns}
\end{frame}




\begin{frame}{La computadora: Un sistema complejo}{CHIP Barcelona}

    \begin{columns}[onlytextwidth,T]
      \column{\dimexpr\linewidth-60mm-5mm}

	\begin{itemize}
	\begin{small}
\bigskip
  \item[Chip] Die en inglés, es empaquetado dentro de un componente que permite su utilizacióin mecánica en un PCB.

\bigskip
\item[Densidad] La tecnología que se utiliza para fabricar los chips (dies), en los circuitos integrados, es el trasistor CMOS.\\
Actualmente la densidad es tan grande que existen miles de millones de transistores en un unico chip.

\item[Barcelona] Un microprocesador de 4 cores

	\end{small}
	\end{itemize}

      \column{50mm}
    \includegraphics[width=80mm]{images/barcelona.png}

    \end{columns}
\end{frame}


\begin{frame}{La computadora: Un sistema complejo}{Transistor CMOS}

    \begin{columns}[onlytextwidth,T]
      \column{\dimexpr\linewidth-60mm-5mm}

	\begin{itemize}
	\begin{small}
\bigskip
  \item[Compuertas] Con transistores CMOS se fabrican compuertas, que pueden operar digitalmente y realizar una simple función lógica.

\bigskip
\item[NOT] Una compuerta NOT se puede fabricar utilizando dos transistores.

	\end{small}
	\end{itemize}

      \column{50mm}
    \includegraphics[width=50mm]{images/cmos2.png}

    \end{columns}
\end{frame}





\begin{frame}{La computadora: Un sistema complejo}

    \begin{columns}[onlytextwidth,T]
      \column{\dimexpr\linewidth-60mm-5mm}

	\begin{itemize}
	\begin{small}
\bigskip
  \item[Abstracción] El hardware y software de una computadora
consiste de una jerarquía en capas, donde cada capa de hardware o software
le oculta detalles a la capa superior.

\bigskip

\item[Principio] \textit{El principio de abstracción} es el que \textit{permite}
a los diseñadores de hardware y software poder \textit{entender la complejidad} 
de los sistemas de cómputo que construyen.

\bigskip

\item [Interfaz] EL nivel \textit{Arquitectura del Conjunto de 
Instrucciones (ISA)}, es la interfaz entre el hardware y el software.

	\end{small}
	\end{itemize}

      \column{60mm}
    \includegraphics[width=50mm]{images/abstracciones2.png}

    \end{columns}

\end{frame}



\begin{frame}{La computadora: Un sistema complejo}{Diseño lógico}

    \begin{columns}[onlytextwidth,T]
      \column{\dimexpr\linewidth-60mm-5mm}

	\begin{itemize}
	\begin{small}
\bigskip
  \item[Diseño digital] o diseño lógico, es actuamente realizado utilizando el algebra de Boole (también llamado algebra de switching)

  \item[SFM] Las máquinas de estado finito pueden ser esquematizadas con diseño digital, y permitan diseñar máquinas algoritmicas.
\bigskip

	\end{small}
	\end{itemize}

      \column{60mm}
    \includegraphics[width=50mm]{images/abstracciones2.png}

    \end{columns}

\end{frame}



%\begin{frame}{La computadora: Un sistema complejo}

%   \begin{columns}[onlytextwidth,T]
%   \column{\dimexpr\linewidth-60mm-5mm}


%  \begin{description}[Snow Storm]
%\begin{frame}{La computadora: Un sistema complejo}

%   \begin{columns}[onlytextwidth,T]
%   \column{\dimexpr\linewidth-60mm-5mm}


%  \begin{description}[Snow Storm]
%  \begin{small}
%  \item[Abstracción] El hardware y software de un procesador
%consiste de una jerarquía en capas, donde cada capa le oculta
%detalles a la capa superior.

%\bigskip

%\item[Principio] Este principio de abstracción es el que permite
%a los diseñadores de hardware y software entender la complejidad 
%de las computadoras.

%\bigskip

%\item [Interfaz] EL nivel \textit{Arquitectura del Conjunto de 
%Instrucciones (ISA)}, es la interfaz entre el hardware y el software.

%\end{small}
%  \end{description}

%    \column{60mm}
%    \includegraphics[width=60mm]{images/abstracciones.png}

%    \end{columns}

%\end{frame}




\subsection{Recursos}

\begin{frame}[fragile]
  \frametitle{Recursos de la materia}

\begin{small}
\begin{itemize}

\item Web: \footnotesize{\texttt http://se.fi.uncoma.edu.ar/ayod1c/}\\
(se alcanza también desde la materia en PEDCO).

\item FOROs de PEDCO (Novedades y Consultas)
\item Telegram (para consultas)
\item Google meet para las exposiciones y discuciones temáticas online\\ (se darán los enlaces de encuentros en las clases).
\item Bibliografía:

\begin{itemize}

\item Andrew S. Tanenbaum (2000), ORGANIZACIÓN DE COMPUTADORAS un enfoque estructurado, Editorial Prentice Hall. (10 copias en biblioteca)
\item David. Patterson, John L. Hennessy, ORGANIZACIÓN Y DISEÑO DE COMPUTADORES La interfaz hardware/software, McGraw-Hill (8 copias en biblioteca).
\item Apuntes y artículos en la web de la materia
\item David. Patterson, John L. Hennessy, Computer Organization and Design RISC-V Edition 1st Edition The Hardware Software Interface. ISBN: 9780128122754

\end{itemize}

\end{itemize}
\end{small}

\end{frame}


\end{document}

%%% Local Variables:
%%% mode: latex
%%% TeX-master: t
%%% TeX-engine: xetex
%%% End:
