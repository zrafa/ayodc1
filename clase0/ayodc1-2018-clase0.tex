\documentclass[glossy]{beamer}
\useoutertheme{wuerzburg}
\useinnertheme[realshadow,corners=2pt,padding=2pt]{chamfered}
\usecolortheme{shark}


\usepackage{listings}
\usepackage[utf8]{inputenc}

\usepackage{tikz}
\newcommand<>{\hover}[1]{\uncover#2{%
 \begin{tikzpicture}[remember picture,overlay]%
 \draw[fill,opacity=0.4] (current page.south west)
 rectangle (current page.north east);
 \node at (current page.center) {#1};
 \end{tikzpicture}}
}

\title{Arquitecturas y Organización de Computadoras I \\\line(1,0){320}}
% \author{\texorpdfstring{Author\newline\url{email@email.com}}{Author}}
%\author{Rafael Ignacio Zurita}
\institute{Rafael Ignacio Zurita \\ Departamento de Ingeniería de Computadoras - FAI - UNCOMA \\ 2018}
%\date{\today}


% \usepackage[spanish, es-tabla]{babel}
\usepackage[utf8]{inputenc}

\begin{document}


\defverbatim[colored]\lstI{
\begin{lstlisting}[language=C++,basicstyle=\ttfamily,keywordstyle=\color{red}]
int main() {
  // Define variables at the beginning
  // of the block, as in C:
  CStash intStash, stringStash;
  int i;
  char* cp;
  ifstream in;
  string line;
[...]
\end{lstlisting}
}


\begin{frame}
\maketitle
\end{frame}

\institute{Departamento de Ingenieria de Computadoras - FAI - UNCOMA \\ 2018}

\begin{frame}
\frametitle{Detalles administrativos}
\textit{A hacker cannot, as they devastatingly put it “approach problem-solving like a plumber in a hardware store”; you have to know what the components actually do.}
 \\
\hfill \hfill How To Become A Hacker -- Eric Steven Raymond
 \\~\\
Docentes
\begin{itemize}
\item Lic. Rafael Ignacio Zurita (rafa@fi.uncoma.edu.ar)
\item An.-Lic. Rodrigo Cañibano (rcanibano@fi.uncoma.edu.ar)
\item Bajo aprobación de la universidad: An. Candelaria Alvarez (candelaria.alvarez@fi.uncoma.edu.ar)
\end{itemize}

 \\~\\
El conocimiento impartido en esta materia, en esta universidad, es en gran medida gracias al profesor Ing. Rodolfo del Castillo (ya jubilado).
\end{frame}



\begin{frame}
 \frametitle{Detalles administrativos}
Horarios
\begin{itemize}
\item  Teoría: martes de 13hs. a 15:30hs. Aula 101 (primer clase 14/08)
\item  Practica: miércoles de 8hs. a 11:30hs. Laboratorios 1 y 3
\end{itemize}
\end{frame}
 
\begin{frame}
 \frametitle{Detalles administrativos}
Fechas Importantes de 2018 (aproximadas)
\begin{itemize}
\item 1er Parcial : 12 de septiembre
\item Recuperatorio 1er Parcial : 26 de septiembre
\item 2do Parcial : 14 de noviembre
\item Recuperatorio 2do Parcial : 28 de noviembre
\item Fecha del primer coloquio:
\item Fecha del segundo coloquio: 
\end{itemize}
Es necesario aprobar los parciales, o sus recuperatorios, para aprobar el cursado de la materia.
Es necesario aprobar cada parcial en primera instancia para acceder a los exámenes para la promoción.
 \\~\\
Los exámenes parciales incluyen únicamente una parte parte práctica.
Los exámenes para la promoción incluyen únicamente temas teóricos.
\end{frame}


\begin{frame}
 \frametitle{Bibliografía}
Libros
\begin{itemize}
\item Andrew S. Tanenbaum (2000), ORGANIZACIÓN DE COMPUTADORAS un enfoque estructurado, Editorial Prentice Hall. (10 copias en biblioteca)
\item David. Patterson John L. Hennessy (1995), ORGANIZACIÓN Y DISEÑO DE COMPUTADORES La interfaz hardware/software, McGraw-Hill (8 copias en biblioteca).
\end{itemize}
Contenido electrónico
\begin{itemize}
\item Apuntes elaborados por la cátedra, disponibles en PEDCO para impresión (pdf) o lectura online (html)
\item Secciones de libros aptas para publicacion
\end{itemize}
\end{frame}

\begin{frame}
 \frametitle{Otros Recursos}
Se encuentra en PEDCO
\begin{itemize}
\item Foros: Novedades y Consultas
\item Programa de la Asignatura
\item Material extra para las prácticas
\end{itemize}
\end{frame}





\begin{frame}
\frametitle{Objetivos y Contenidos Mínimos}

Que el alumno logre: comprender la estructura interna de una computadora a nivel de análisis de 
circuitos digitales. Comprender la representación de datos e instrucciones a nivel máquina. 
Comprender los conceptos de programación en lenguaje ensamblador.

 \\~\\

Representación de datos a nivel máquina. Jerarquía de memoria. Organización funcional. Circuitos
combinatorios y secuenciales. Lenguaje Ensamblador. Conceptos de representación en punto flotante
y de error. Conceptos de Máquinas Algorítmicas, Procesadores de alta prestación, Arquitecturas no
Von Neumann y Arquitecturas Reconfigurables.
\end{frame}

\begin{frame}
 \frametitle{Consejos y preguntas}
\begin{center}
\begin{itemize}
\item  ¿Preguntas?
\end{itemize}
\end{center}
\end{frame}
\end{document}
