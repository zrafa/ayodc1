\documentclass[glossy]{beamer}
\useoutertheme{wuerzburg}
\useinnertheme[realshadow,corners=2pt,padding=2pt]{chamfered}
\usecolortheme{shark}

\usepackage{listings}
\usepackage[utf8]{inputenc}

\usepackage{tikz}
\newcommand<>{\hover}[1]{\uncover#2{%
 \begin{tikzpicture}[remember picture,overlay]%
 \draw[fill,opacity=0.4] (current page.south west)
 rectangle (current page.north east);
 \node at (current page.center) {#1};
 \end{tikzpicture}}
}

\title{Arquitecturas y Organización de Computadoras I \\\line(1,0){320}}
% \author{\texorpdfstring{Author\newline\url{email@email.com}}{Author}}
%\author{Rafael Ignacio Zurita}
\institute{Rafael Ignacio Zurita \\ Departamento de Ingenieria de Computadoras - FAI - UNCOMA 2018 \\ Clase presencial 3}
%\date{\today}



\begin{document}




\begin{frame}
\maketitle
\end{frame}

\institute{Departamento de Ingenieria de Computadoras - FAI - UNCOMA \\ 2018}

\begin{frame}
\frametitle{Programa Analítico}
\textbf{UNIDAD I: Arquitectura y Organización de Computadoras}
 \\~\\
\textit{Introducción al lenguaje ensamblador MIPS. Representación de datos a nivel de máquina. Direccionamiento de memoria: concepto de palabra, ordenamiento de bytes. Registros.} 
 \\~\\
\end{frame}


\begin{frame}
\frametitle{Introducción a MIPS}
\begin{center}
\begin{figure}
\includegraphics[scale=0.4]{organizacion-mips.jpg} 
\caption{Diagrama de Bloques de una computadoroa MIPS}
\label{Diagrama de Bloques de una computadoroa MIPS}
\end{figure}
\end{center}
\end{frame}


\begin{frame}
\frametitle{Representación de datos a nivel máquina}
\textbf{Organización de la Memoria principal en MIPS}
\begin{itemize}
	\item $2^{32}$ bytes, con direcciones 0, 1, 2 ... 
	\item $2^{30}$ 4-bytes palabras, con direcciones 0, 4, 8 ... 
	\item Las direcciones de las palabras deben ser múltiplos de 4
\end{itemize}
	\begin{center}
\includegraphics[scale=0.5]{memoria-mips.jpg} 
	\end{center}
\end{frame}





% INDICE: HISTORIA de procesadores y arquitectura (filmina de arquitectura de david), system 360, zseries, 
% compilacion, ensamblado, codigo maquina
% ejemplo x86 , uso de longitud variable para instruccciones, instrucciongs con dos operandos, pocos registros, infinidad de modos de direccion2miento para cada operacion. En cada version nuevas instrucciones y compatibilidad. Cientos de instrucciones CISC
% Microarquitectura : programada (interprete), en hardware
% grafico inicio de RISC: quitar el interprete, hacer un set sencillo, fijo y de pocos formatos
% limite de energia, nacimiento de arquitecturas avanzadas 
%
% lenguaje ensamblador, figura de compilacion, ejemplo de ensamblador x86, figura de dos arquitecturas x86, system 360, zseries,X , filmina de diferencia entre arquitectura y organizacion, estructura por capas y resumen de los temas a ver en la materia,  memoria, 
% segunda parte: antes de ver lenguaje ensamblador veremos terminologia importante sobre la organizacion  de la memoria
% Representación de datos a nivel de máquina. Direccionamiento de memoria: concepto de palabra, ordenamiento de bytes. Registros.} 
% memori2, direccion, byte, palabra, alineacion, endianess, instrucciones y datos
% directivas MIPS, etc para el practico

\begin{frame}
\frametitle{Programación de la máquina: lenguaje ensamblador MIPS}
        \begin{center}
        \textbf{Conjunto básico de instrucciones MIPS}
        \end{center}
\begin{tabular}{cl}

\begin{tabular}{c}
\includegraphics[height=5cm, width=4cm]{instrucciones-mips.jpg}

\end{tabular}
& \begin{tabular}{l}
\parbox{0.5\linewidth}{

\includegraphics[height=5.2cm, width=4cm]{pseudoinstrucciones-mips2.jpg}
}
\end{tabular} \\

\end{tabular}
\end{frame}


\begin{frame}
\frametitle{Formato de  instrucciones en MIPS}
	\textbf{3 Formatos (fácil decodificación en hardware)}
	\begin{center}
\includegraphics[scale=0.5]{formato.jpg} 
	\end{center}
\end{frame}


\begin{frame}
\frametitle{Formato de instrucciones en MIPS}
	\textbf{Un ejemplo completo de traducción al lenguaje de la máquina}
	\begin{center}
\includegraphics[scale=0.4]{below2.jpg} 
	\end{center}
\end{frame}


\begin{frame}
 \frametitle{Consejos y preguntas}
\begin{center}
\begin{itemize}
\item  ¿Preguntas?
\end{itemize}
\end{center}
\end{frame}


\begin{frame}
 \frametitle{Bibliografía}
Libros
\begin{itemize}
\item Andrew S. Tanenbaum (2000), ORGANIZACIÓN DE COMPUTADORAS un enfoque estructurado, Editorial Prentice Hall. (10 copias en biblioteca)
\item David. Patterson John L. Hennessy (1995), ORGANIZACIÓN Y DISEÑO DE COMPUTADORES La interfaz hardware/software, McGraw-Hill (8 copias en biblioteca).
\end{itemize}
Contenido electrónico
\begin{itemize}
	\item \textbf{x86 assembly basis} Una introducción al lenguaje ensamblador x86. Disponible en PEDCO en formato PDF.
		\url{https://www.nayuki.io/page/a-fundamental-introduction-to-x86-assembly-programming}
\item Apuntes elaborados por la cátedra, disponibles en PEDCO para impresión (pdf) o lectura online (html)
\item Secciones de libros aptas para publicacion
\end{itemize}
\end{frame}


\end{document}
