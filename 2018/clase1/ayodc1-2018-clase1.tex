\documentclass[glossy]{beamer}
\useoutertheme{wuerzburg}
\useinnertheme[realshadow,corners=2pt,padding=2pt]{chamfered}
\usecolortheme{shark}

\usepackage{listings}
\usepackage[utf8]{inputenc}

\usepackage{tikz}
\newcommand<>{\hover}[1]{\uncover#2{%
 \begin{tikzpicture}[remember picture,overlay]%
 \draw[fill,opacity=0.4] (current page.south west)
 rectangle (current page.north east);
 \node at (current page.center) {#1};
 \end{tikzpicture}}
}

\title{Arquitecturas y Organización de Computadoras I \\\line(1,0){320}}
% \author{\texorpdfstring{Author\newline\url{email@email.com}}{Author}}
%\author{Rafael Ignacio Zurita}
\institute{Rafael Ignacio Zurita \\ Departamento de Ingenieria de Computadoras - FAI - UNCOMA 2018 \\ Clase presencial 1}
%\date{\today}

\begin{document}




\begin{frame}
\maketitle
\end{frame}

\institute{Departamento de Ingenieria de Computadoras - FAI - UNCOMA \\ 2018}

\begin{frame}
\frametitle{Programa Analítico}
\textbf{UNIDAD I: Arquitectura y Organización de Computadoras}
 \\~\\
\textit{Organización  funcional. Repaso del modelo de Von Neumann. Concepto de Arquitectura y Organización de Computadoras. Representación de datos a nivel de máquina. Direccionamiento de memoria: concepto de palabra, ordenamiento de bytes. Registros.} 
 \\~\\
Formatos de Instrucciones. Modos de direccionamiento. Tipos de instrucciones: transferencia de datos, operaciones aritméticas y lógicas, transferencia de control. Excepciones. Lenguaje ensamblador: directivas, operaciones, pseudo-operaciones, macros.
\end{frame}

\begin{frame}
\frametitle{Introducción}
\textit{A hacker cannot, as they devastatingly put it “approach problem-solving like a plumber in a hardware store”; you have to know what the components actually do.}
 \\
\hfill \hfill How To Become A Hacker -- Eric Steven Raymond
 \\~\\
\textbf{Concepto de Arquitectura y Organización de una Computadora}
\begin{itemize}
\item ¿Qué significa Arquitectura?
\item ¿Qué significa Organización?
\item Pero antes: ¿Qué es una computadora?
\item Repaso: Modelo de Von Neumann
\end{itemize}

\end{frame}



\begin{frame}
\frametitle{Computadora}
Programa almacenado
\begin{figure}
\includegraphics[scale=0.5]{maq-von-neumann.jpg} 
\caption{Modelo de Von Neumann}
\label{Modelo de Von Neumann}
\end{figure}
 ¿programa-almacenado?
\end{frame}
 
\begin{frame}
\frametitle{Computadora ENIAC}
Arquitectura anterior a Programa almacenado
\begin{figure}
\includegraphics[scale=0.4]{eniac.jpg} 
\caption{ENIAC}
\label{ENIAC}
\end{figure}
Se demoraban varias semanas en programar.
\end{frame}
 
\begin{frame}
\frametitle{Computadora ENIAC}
Arquitectura anterior a Programa almacenado
\begin{figure}
\includegraphics[scale=0.4]{eniac2.jpg} 
\caption{ENIAC 2}
\label{ENIAC 2}
\end{figure}
Se demoraban varias semanas en programar.
\end{frame}
 
 
\begin{frame}
\frametitle{Organización de una computadora}
\begin{itemize}
\item CPU: contiene REGISTROS, ALU, Unidad de Control
\item Memoria
\item Dispositivos de E/S
\item Todo interconectado mediante buses
\end{itemize}
\end{frame}

\begin{frame}
\frametitle{Arquitectura de computadoras}
Especificación utilizada mayormente por el programador
\begin{figure}
\includegraphics[scale=0.3]{milestones.jpg} 
%\caption{Organización de una computadora}
%\label{Organización de una computadora}
\end{figure}
"Lo que el programador reconoce y utiliza de una computadora en el nivel mas bajo posible"
\end{frame}
 
\begin{frame}
\frametitle{Organización de computadora estructurada}
Estructura por Niveles
\begin{itemize}
\item Es una manera de esquematizar los conceptos por niveles
\item En los niveles mas bajos está el hardware
\item En los mas altos el software
\end{itemize}
\end{frame}

\begin{frame}
\frametitle{Organización estructurada por niveles}
Segun Tanenbaum
\begin{figure}
\includegraphics[scale=0.4]{seis-niveles.jpg} 
\caption{Maquina multinivel}
\label{Maquina multinivel}
\end{figure}
\end{frame}
 
\begin{frame}
\frametitle{Organización estructurada por niveles}
Otra visión (Harris)
\begin{figure}
\includegraphics[scale=0.4]{niveles-abstraccion.jpg} 
\caption{Maquina multinivel}
\label{Maquina multinivel}
\end{figure}
\end{frame}
 

\begin{frame}
\frametitle{Organización de computadora estructurada}
\begin{itemize}
\item Lenguaje de alto nivel
\item Lenguaje ensamblador
\item ISA (Arquitectura)
\item CPU - Memoria - E/S - Buses (Organización/Microarquitectura)
\item circuitos secuenciales, combinacionales, MSF, IC
\item diseño lógico
\item compuertas (puertas). NOR, NAND, etc
\item transistores (CMOS)
\item física de los elementos
\end{itemize}
\end{frame}

\begin{frame}
\frametitle{Arquitecturas y Organización de Computadoras}
¿Por qué estudiar todos estos conceptos?
\begin{itemize}
\item Tiempo de Ejecución de programas
\item Control de riego / laboratorio
\item Prestaciones - Multiprocesadores - Arquitecturas avanzadas - Consumo
\end{itemize}
\begin{itemize}
\item Tanque de agua para riego (¿hay vidas en riesgo?)
\end{itemize}
\end{frame}

\begin{frame}
\frametitle{Arquitecturas y Organización de Computadoras}
Clases de computadoras
\begin{itemize}
\item PC - workstation
\item Servidor
\item Cluster
\item SuperComputadora (cientos de miles de procesadores, terabytes, petabytes)
\item Sistemas embebidos
\end{itemize}
\end{frame}
 
\begin{frame}
 \frametitle{Bibliografía}
Libros
\begin{itemize}
\item Andrew S. Tanenbaum (2000), ORGANIZACIÓN DE COMPUTADORAS un enfoque estructurado, Editorial Prentice Hall. (10 copias en biblioteca)
\item David. Patterson John L. Hennessy (1995), ORGANIZACIÓN Y DISEÑO DE COMPUTADORES La interfaz hardware/software, McGraw-Hill (8 copias en biblioteca).
\end{itemize}
Contenido electrónico
\begin{itemize}
\item Apuntes elaborados por la cátedra, disponibles en PEDCO para impresión (pdf) o lectura online (html)
\item Secciones de libros aptas para publicacion
\end{itemize}
\end{frame}

\begin{frame}
 \frametitle{Otros Recursos}
Se encuentra en PEDCO
\begin{itemize}
\item Foros: Novedades y Consultas
\item Programa de la Asignatura
\item Material extra para las prácticas
\end{itemize}
\end{frame}







\begin{frame}
 \frametitle{Consejos y preguntas}
\begin{center}
\begin{itemize}
\item  ¿Preguntas?
\end{itemize}
\end{center}
\end{frame}
\end{document}
